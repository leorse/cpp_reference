%----------------------------------------------------------
%
\documentclass[onecolumn]{book}%

\usepackage{amsmath}%
\usepackage{amsfonts}%
\usepackage{amssymb}%
\usepackage{graphicx}
\usepackage[latin1]{inputenc}
\usepackage[french]{babel}
\usepackage{fontspec}
\usepackage{makeidx}
\usepackage{wrapfig}
\usepackage{hyperref}
\usepackage{color}
\usepackage{listings}
\usepackage{xcolor}
\usepackage{textcomp}
\usepackage[T1]{fontenc}
%\usepackage{soul}
\usepackage{framed} % for contrast
\newcommand{\image}[1]{\hl{IMAGE #1}}
\newcommand{\imagecap}[2]{
\begin{figure}
    \includegraphics{../images/image#1}
    \caption{#2}
		\label{fig:#1}
\end{figure}
}
\newcommand{\imagenocap}[1]{\includegraphics{../images/image#1}}

\newcommand{\newdef}[1]{#1\index{#1@\textsc{#1}}}
\newcommand{\newdefDouble}[2]{#1\index{#2@\textsc{#1}}}
\newcommand{\newdefDoubleSE}[3]{#2\index{#2@\textsc{#1}!#3}}

\lstset{
language=C++,
basicstyle=\ttfamily\scriptsize, % ou ça==> basicstyle=\scriptsize,
upquote=true,
%aboveskip={1.5\baselineskip},
columns=fullflexible,
showstringspaces=false,
escapeinside={\%*}{*)},
numbers=left,
extendedchars=true,
breaklines=true,
showtabs=false,
showspaces=false,
showstringspaces=false,
identifierstyle=\ttfamily,
keywordstyle=\color[rgb]{0,0,1},
commentstyle=\color[rgb]{0.133,0.545,0.133},
stringstyle=\color[rgb]{0.627,0.126,0.941},
}

%----------------------------------------------------------

%----------------------------------------------------------
\makeindex
\begin{document}

\title{R\'ef\'erences C++}
\author{Damien MESSNER}
\maketitle
\tableofcontents

\chapter{Les nouveaut\'es du C++ par rapport au C}
\section{const}
Parlons du mot cl\'e \emph{const}\index{const@\textsc{mots cl\'es}!const} qui permet de d\'efinir une variable comme \'etant constante, un peu comme les macros en C.\\

Une fois qu'une variable est d\'efinie comme constante, alors elle ne peut plus changer. Elle ne peut \^etre initialis\'ee qu'à la d\'eclaration:
\begin{framed}
\begin{lstlisting}
int const val = 3; //initialisation à la d\'eclaration
val = 5; //%*\textbf{\textcolor{red}{error}}*): assignment of read-only variable 'val'
\end{lstlisting}
\end{framed}

\'etant constante, elle peut \^etre utilis\'ee pour d\'efinir la taille d'un tableau:

\begin{framed}
\begin{lstlisting}
char tab[val] = "AB"; //utilisation de const pour d\'eterminer la taille d'un tableau
\end{lstlisting}
\end{framed}

Le mot cl\'e \emph{const} peut aussi \^etre utilis\'e pour les pointeurs. On d\'etermine alors un pointeur pouvant \^etre modifi\'e ou non, ou alors d\'eterminer si le contenu vers lequel il pointe peut \^etre modifi\'e.
Dans l'exemple qui suit, le contenu est d\'eclar\'e constant, le pointeur peut quant à lui \^etre modifi\'e.
\begin{framed}
\begin{lstlisting}
char const *buffer; //le pointeur peut changer, mais pas le contenu vers lequel il pointe
buffer = tab; //autoris\'e car le pointeur peut changer
*buffer = 'C'; //%*\textbf{\textcolor{red}{error}}*): assignment of read-only location '* buffer'
\end{lstlisting}
\end{framed}
Dans l'exemple suivant, c'est l'inverse, le pointeur ne peut \^etre modifi\'e, mais son contenu le peut. Le pointeur doit alors \^etre obligatoirement \^etre initialis\'e à sa d\'eclaration:
\begin{framed}
\begin{lstlisting}
char *const tampon = tab; //le pointeur ne peut plus changer, mais son contenu oui
tampon++; //%*\textbf{\textcolor{red}{error}}*): increment of read-only variable 'tampon'
*tampon = 'C'; //on peut modifier le contenu
\end{lstlisting}
\end{framed}
De m\^eme qu'il est possible alors de d\'efinir le pointeur comme \'etant constant ainsi que son contenu:
\begin{framed}
\begin{lstlisting}
char const *const zone = tab; //on ne peut ni modifier le pointeur, ni le contenu
zone++; //%*\textbf{\textcolor{red}{error}}*): increment of read-only variable 'zone'
*zone = 'C'; //%*\textbf{\textcolor{red}{error}}*): assignment of read-only location '*(const char*)zone'
\end{lstlisting}
\end{framed}

Dans une classe on peut se servir de \emph{\newdefDoubleSE{const}{const}{sur m\'ethode}} pour rendre constant les propri\'et\'es d'une classe dans une m\'ethode de cette m\^eme classe.
La m\'ethode est d\'efinie \emph{const}, ainsi toutes les propri\'et\'es de cette m\^eme classes sont alors en lecture seule:
\begin{framed}
\begin{lstlisting}
class constClass
{
private:
  int une_var;
public:
constClass();
int constFonc(int) const; //d\'efinition de la m\'ethode const
};

int constClass::constFonc(int) const
{
  return this->une_var;   //%*\textbf{\textcolor{green}{OK}}*): pas de modification de la propri\'et\'e une_var
  return this->une_var++; //%*\textbf{\textcolor{red}{error}}*): increment of member 'constClass::une_var' in read-only object
}
\end{lstlisting}
\end{framed}

La règle pour savoir ce qui est constant de ce qui ne l'est pas est de ce dire:
\begin{quotation}
Ce qui se trouve à gauche est constant
\end{quotation}

\section{op\'erateur ::}
Cet op\'erateur est utile, entre autres, quand une variable globale a le m\^eme nom qu'une variable locale. L'op\'erateur \newdef{::} permet alors de pr\'eciser la port\'ee globale.
\begin{framed}
\begin{lstlisting}
int une_var_globale = 12;
une_var_globale = 50; //variable locale
::une_var_globale = 25; //variable globale
\end{lstlisting}
\end{framed}

\section{les r\'ef\'erences}
Une r\'ef\'erence est comme son nom l'indique, une \newdefDouble{r\'ef\'erence}{reference} directe vers une autre variable. Elle ne contient pas une valeur, mais la r\'ef\'erence d'une autre variable.
Elle doit \^etre initialis\'ee que depuis une autre variable, et cela ne peux pas changer.
\begin{framed}
\begin{lstlisting}
int une_var = 45;
int &varValue = 12; //%*\textbf{\textcolor{red}{error}}*): invalid initialization of non-const reference of type 'int&' from an rvalue of type 'int'
int &varNoInit; //%*\textbf{\textcolor{red}{error}}*): 'var' declared as reference but not initialized
int &var = une_var; //initialisation correcte
\end{lstlisting}
\end{framed}
On peut utiliser le mot cl\'e \emph{\newdefDoubleSE{const}{const}{sur r\'ef\'erence}} avec les r\'ef\'erences:
\begin{framed}
\begin{lstlisting}
int const &constVar = une_var; //rf\'erence constante vers une_var
var++; //incr\'ement possible
constVar++; //%*\textbf{\textcolor{red}{error}}*): increment of read-only reference ?constVar?
\end{lstlisting}
\end{framed}
Elles sont int\'eressantes pour manipuler des objets larges entres des fonctions. Plutôt que de copier en m\'emoire des objets volumineux, on pr\'ef\'erera les passer par r\'ef\'erence:
\begin{framed}
\begin{lstlisting}
struct monObjet
{
  char chaine[50];
};

monObjet const & testReference(monObjet const &obj)
{
  return obj; //la r\'ef\'erence, non une copie, est pass\'ee en paramètre
}
\end{lstlisting}
\end{framed}
\section{R-value}
Il semblerait que les \newdef{r-value} ne fonctionnent que depuis C++11.
\section{Les templates}
Un \newdef{template} permet de construire une classe ou une m\'ethode sans mentionner le type et ainsi les rendre g\'en\'eriques.
La construction pour une classe est la suivante:
\begin{framed}
\begin{lstlisting}
template<typename T>
class monTemplate
{
public:
  T monType;
monTemplate(const T& val = T());
~monTemplate();
};

template<typename T>
monTemplate<T>::monTemplate(const T& unType) : monType(unType)
{
  std::cout << "mon constructeur:" << monType << std::endl;
}


template<typename T>
monTemplate<T>::~monTemplate()
{
  std::cout << "mon destructeur" << std::endl;
}
\end{lstlisting}
\end{framed}
\begin{itemize}
\item Pour des raisons pratiques, il est pr\'ef\'erable de d\'eclarer le tout directement dans un fichier d'ent\^ete .h.
\item Lors de la d\'efinition d'un template, le mot cl\'e utilis\'e pour identifier le type peut \^etre \emph{typename} ou alors \emph{class}. Les deux ont la m\^eme signification.
\end{itemize}
Un template peut avoir plusieurs paramètre:
\begin{framed}
\begin{lstlisting}
template <typename T, typename U, typename V>
std::string retourAffiche(T un, U deux, V trois)
{
 /*  */
 return str;
}
\end{lstlisting}
\end{framed}
Il est possible d'appeler une fonction template depuis une autre fonction template:
\begin{framed}
\begin{lstlisting}
template <typename T>
std::string convertMe(T aconvertir)
{
  std::ostringstream oss;
  
  oss << aconvertir;
  return oss.str();
}

//3 paramètres
template <typename T, typename U, typename V>
std::string retourAffiche(T un, U deux, V trois)
{
  std::string str = "";
  str += un;
  str += convertMe<U>(deux); //on appelle la m\'ethode convertMe en pr\'ecisant le type U, car on ne sait pas à l'avance quel type il sera
  str += convertMe<V>(trois); // m\^eme chose
  
  return str;
}
\end{lstlisting}
\end{framed}
\newdefDoubleSE{const, mot cl\'e}{const}{sur r\'ef\'erence}
Il est facile d'utiliser l'op\'erateur \newdef{<<} avec un \emph{\newdefDoubleSE{ostringstream}{ostringstream}{conversion de type}} afin d'effectuer une conversion de type\index{m\'ethodes@\textsc{m\'ethodes}!conversion de type}:
\begin{framed}
\begin{lstlisting}
//m\'ethode de conversion d'un type vers une chaine
template <typename T>
std::string convertMe(T aconvertir)
{
  std::ostringstream oss;
  //flux depuis un type donn\'ee, vers un std::string
  oss << aconvertir;
  return oss.str();
}
\end{lstlisting}
\end{framed}
Un typename peut aussi \^etre utilis\'e en retour:
\begin{framed}
\begin{lstlisting}
//m\'ethode qui va convertir un type vers un autre du moment que les deux savent utiliser l'op\'erateur >> ou <<
template <typename T, typename V>
V convertMe(T aconvertir)
{
  V destination;
  std::istringstream iss(convertMe<T>(aconvertir) );
  iss >> destination;
  
  return destination;
}
\end{lstlisting}
\end{framed}
Cette m\'ethode permet de faire une conversion de deux types diff\'erents \index{m\'ethodes@\textsc{m\'ethodes}!conversion de deux types diff\'erents} à l'aide de la classe \newdefDoubleSE{istringstream}{istringstream}{conversion de type}.
\section{auto}
\emph{Disponible à partir de C++11}
L'op\'erateur \newdef{auto} permet une meilleure lisibilit\'e de code en simplifiant l'\'ecriture des type complexe. LE travail revient au compilateur qui d\'etermine lui m\^eme type en faisant de l'inf\'erence de type.
\begin{framed}
\begin{lstlisting}
auto sortie  = convertMe<std::string, float>("123.654"); //la fonction convertMe renvoie un float, auto sera alors un float
std::cout << "auto sortie:" << sortie << std::endl;
\end{lstlisting}
\end{framed}
Son utilisation est surtout pratique lors des it\'erations\index{iterator@\textsc{iterator}!avec auto}. Supposons qu'on ait la liste suivante:
\begin{framed}
\begin{lstlisting}
std::list<int> liste;
  
liste.push_back(10);
liste.push_back(20);
liste.push_back(56);
\end{lstlisting}
\end{framed}
La parcourir avec la m\'ethode C++98 se r\'ev\'elerait à ceci:
\begin{framed}
\begin{lstlisting}
//avec toute la d\'efinition
std::list<int>::iterator it = liste.begin();
while(it != liste.end())
{
    std::cout << "elem:" << *it << std::endl;
    it++;
}
\end{lstlisting}
\end{framed}
Ici, on a une liste assez simple et la d\'efinition de l'it\'erateur est d\'ejà compliqu\'e. Avec l'op\'erateur \emph{auto} ça donne ceci:
\begin{framed}
\begin{lstlisting}
//avec auto
auto it = liste.begin();
while(it != liste.end())
{
    std::cout << "elem:" << *it << std::endl;
    it++;
}
\end{lstlisting}
\end{framed}
Cette fois-ci, c'est le compilateur qui se charge de d\'eterminer la type de l'it\'erateur. Et le code est beaucoup plus lisible.
\chapter{Le c++ dans toute sa splendeur}
\section{Initialisation avec accolade}
L'initialisation d'une variable ou d'une classe peut se faire avec les parenthèses, avec une initialisation par valeur ou alors avec des accolades.\\
Si elles permettent des initialisations, elles servent aussi pour l'initialisation de listes.
\begin{framed}
\begin{lstlisting}
//les d\'eclarations suivantes ont le m\^eme r\'esultat
int i = 20; //d\'eclaration avec initilisation
int j(30);  //m\^eme chose mais avec des parenthèses
int k{40};  //m\^eme chose mais avec des accolades
\end{lstlisting}
\end{framed}
Il est possible d'utiliser la m\^eme syntaxe dans une classe pour les membres non statiques. Mais dans ce cas là, les d\'eclarations avec parenthèse sont interdites:
\begin{framed}
\begin{lstlisting}
class c7Class1
{
private:
    int c1=15;     //OK
    int c2{25};    //OK
    //int c3(35);  //KO: pas de parenthèses
    
    ...
    
};
\end{lstlisting}
\end{framed}
Si on rajoute les \'el\'ements suivants à la classe pr\'ec\'edente:
\begin{framed}
\begin{lstlisting}
public:
    c7Class1() {}
    c7Class1(int val1, int val2):c1(val1),c2(val2)
    {
        
    }
    void printContenu(void)
    {
        cout<<"c1:"<<c1<<", c2:"<<c2<<endl;
    }
\end{lstlisting}
\end{framed}
Et que l'on instancie de cette manière:
\begin{framed}
\begin{lstlisting}
//d\'eclaration avec constructeur par d\'efaut
c7Class1* c7class1 = new c7Class1();
c7class1->printContenu();  //c1:15, c2:25
//d\'eclaration avec autre constructeur
c7Class1* c7class2 = new c7Class1(6, 7);
c7class2->printContenu();  //c1:6, c2:7
//d\'eclaration sans new
c7Class1 c7class3(45, 35);
c7class3.printContenu();   //c1:45, c2:35
//d\'eclaration avec accolades
c7Class1 c7class4{87, 97};
c7class4.printContenu();   //c1:87, c2:97
\end{lstlisting}
\end{framed}
On constate alors que chacune de ces d\'eclarations est valable.\\
Mais alors pourquoi utiliser une syntaxe suppl\'ementaire?\\
Pour deux raisons:
\begin{itemize}
\item Interdire la conversion implicite
\item Permettre l'initialisation par liste
\end{itemize}
Pour la conversion implicite:
\begin{framed}
\begin{lstlisting}
//d\'eclaration avec double
c7Class1 c7class5(45.5f, 85.5f); //appel du constructure (int, int) alors que l'on passe des doubles OK
c7class5.printContenu();
//d\'eclaration avec double mais entre accolade
//c7Class1 c7class6{12.0, 42.5}; //%*\textbf{\textcolor{red}{error}}*), pas de conversion possible entre double et int dans ce cas là
\end{lstlisting}
\end{framed}
Pour l'initialisation par liste:
\begin{framed}
\begin{lstlisting}
//Dans la ligne suivante, on cr\'e\'e un vecteur avec 3 \'el\'ements entier
vector<int> vint{1, 2, 5};
\end{lstlisting}
\end{framed}
\section{Utilisation de nullptr au lieu de NULL}
\index{nullptr@\textsc{mots cl\'es}!nullptr}
Il est fr\'equent d'utiliser NULL pour identifier un pointeur nul. Cependant la conversion de type dans ces cas là peut \'echouer et mal induire le compilateur.\\
Consid\'erons les fonctions suivantes:
\begin{framed}
\begin{lstlisting}
void fonc(int)
{
    cout<<"fonc int"<<endl;
}

void fonc(bool)
{
    cout<<"fonc bool"<<endl;
}

void fonc(void*)
{
    cout<<"fonc void*"<<endl;
}
\end{lstlisting}
\end{framed}
Ceux sont toutes les m\^emes surcharg\'ees avec un type diff\'erent.\\
Voici ce qu'on obtient avec les appels suivants:
\begin{framed}
\begin{lstlisting}
fonc(0);        //OK: appel fonc(int)
fonc(NULL);     //KO: appel fonc(int)
fonc((void*)0); //OK: appel fonc(void*)
fonc(nullptr);  //OK: appel fonc(void*)
\end{lstlisting}
\end{framed}
On constate que lorsque l'on souhaitait appeler syst\'ematiquement les fonctions "void*", cela n'a jamais \'et\'e le cas.\\
Maintenant plus grave, ce qui ne provoque qu'un warning jusque là provoque une erreur si l'on passe par un template:
\begin{framed}
\begin{lstlisting}
//cette fonction attend un pointeur int* et rien d'autre
void fonc_ptr(int* ptr)
{
    if(ptr == nullptr)
    {
        cout<<"pointeur nul"<<endl;
    }
    else
    {
        cout<<"valeur:"<<*ptr<<endl;
    }
    
}

//le template va appeler cette fonction
template<typename PTR>
void template_fonc(PTR ptr)
{
    fonc_ptr(ptr);
}
\end{lstlisting}
\end{framed}
Et maintenant avec les appels suivants:
\begin{framed}
\begin{lstlisting}
//notre variable que l'on va afficher
int val=45;
template_fonc(0);       //%*\textbf{\textcolor{red}{error}}*): Le template transmet un entier, incompatible avec fonc_ptr
template_fonc(NULL);    //%*\textbf{\textcolor{red}{error}}*): Exactement le m\^eme problème
template_fonc(nullptr); //OK: nullptr est convertible en n'importe quel pointeur, affiche "pointeur nul"
template_fonc(&val);    //OK: on transmet bien un pointeur sur un entier, affiche "45"
\end{lstlisting}
\end{framed}
\chapter{La STL}
\section{Les it\'erateurs}
\index{iterator@\textsc{mots cl\'es}!iterator}
\section{les threads}
voir ce lien plus tard: https://solarianprogrammer.com/2012/02/27/cpp-11-thread-tutorial-part-2/?ref=dzone
\printindex

\end{document}
